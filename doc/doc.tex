\documentclass[11pt,a4paper]{article}

\usepackage[left=2cm,text={17cm,24cm},top=3cm]{geometry}
\usepackage[english]{babel}
\usepackage[utf8]{inputenc}
\usepackage[T1]{fontenc}

\usepackage{url}
\usepackage{tikz}
\usepackage{float}
\usepackage{xcolor}
\usepackage{siunitx}
\usepackage{amsmath}
\usepackage{accents}
\usepackage{comment}
\usepackage{listings}
\usepackage{csquotes}
\usepackage{hyperref}
\usepackage{textcomp}
\usepackage{amsfonts}
\usepackage{breakurl}
\usepackage{etoolbox}
\usepackage{graphicx}
\usepackage{multicol}
\usepackage{multirow}
\usepackage{indentfirst}
\usepackage{supertabular}
\usepackage[titles]{tocloft}

\def\UrlBreaks{\do\/\do-} % URL breaking characters

\newcommand{\red}[1]{\textcolor{red}{#1}} % \red{text in red}
\newcommand{\blue}[1]{\textcolor{blue}{#1}} % \blue{text in blue}
\newcommand{\TODO}{\textbf{\textcolor{red}{TODO}}} % red bold TODO
\newcommand{\tilda}{\raisebox{0.5ex}{\texttildelow}} % command \tilda for '~' character

\renewcommand{\cftdot}{}

\setlength\parindent{0pt} % do NOT indent
\graphicspath{{img/}} % path to images

\patchcmd{\thebibliography}{\section*{\refname}}{}{}{}

\begin{document}

\begin{titlepage}

    \begin{center}
        % FIX: lines must end with '%', if not then it will result in an incorrect centering
        \vfill {%
            \Huge{%
                \textsc{%
                    Faculty of Informatics\\[3mm]%
                    Masaryk University%
                }%
            }%
        }%

        \hfill\\[15mm]

        \begin{figure}[!h]
            \centering
            \includegraphics[scale=3]{muni-fi-logo.pdf}
        \end{figure}

        \hfill\\[10mm]

        \Huge{
            \textbf{
                PA181
            }
        }

        \hfill\\[-10mm]

        \huge{
            \textbf{
                Services - Systems, Modeling and Execution
            }
        }

        \hfill\\[10mm]

        \LARGE{
            \textbf{
                Term Project Documentation
            }
        }
        \vfill

    \end{center}

        \Large{
            \hfill\\
            Adrián Tóth (491322)\\
            Jiří Čechák (445717)\\
            Jan Ondruch (433341)\\
            Tadeáš Pavlík (487555)\\
            Václav Stehlík (487580) \hfill \today
        }

\end{titlepage}

\setlength{\parskip}{0pt}
    \hypersetup{hidelinks}\tableofcontents
\setlength{\parskip}{0pt}

\newpage

\section{About}

Term project for course \textit{PA181 Services - Systems, Modeling and Execution}\footnote{\href{https://is.muni.cz/predmet/fi/jaro2019/PA181}{is.muni.cz/predmet/fi/jaro2019/PA181}} in year 2019. Within the project, we had to create a fully functional application using the \textit{IBM Cloud}\footnote{\href{https://cloud.ibm.com/}{cloud.ibm.com}} technology including a detailed documentation and a presentation. \textit{doc. Mouzhi Ge, Ph.D.}\footnote{\href{https://is.muni.cz/auth/osoba/239833}{is.muni.cz/auth/osoba/239833}} is the project supervisor.

\section{Idea}

The core idea was to create and develop a useful and practical application. The application provide services for testing the users in a form of questions and answers. Users are able to test themselves via these questions by selecting the correct answers. There are severals tests in three different types of language (Czech, Slovak and English).

\section{Used Technologies}

The following technologies were integrated and used during the development process:
\begin{itemize}
    \item cloud based platform
    \begin{itemize}
        \item IBM Cloud\footnote{\href{https://cloud.ibm.com/}{cloud.ibm.com}}
    \end{itemize}

    \item version control system (VCS)
    \begin{itemize}
        \item GitHub\footnote{\href{https://github.com/}{github.com}}
    \end{itemize}

    \item continuous integration (CI)
    \begin{itemize}
        \item Travis CI\footnote{\href{https://travis-ci.org/}{travis-ci.org}}
    \end{itemize}
\end{itemize}

\section{Work Division}

Our team consisted of 5 members: Adrián Tóth, Jan Ondruch, Jiří Čechák, Tadeáš Pavlík and Václav Stehlík.\\

Everyone from us was in charge of a certain part of the project. The work was divided as the following:
\begin{itemize}
    \item Adrián Tóth
    \begin{itemize}
        \item project initialization
        \item VCS initialization
        \item creation of project skeleton
        \item Travis CI integration and configuration
        \item IBM DevOps toolchain configuration
        \item creation of continuous delivery pipeline
        \item project deployment
    \end{itemize}

    \item Jan Ondruch
    \begin{itemize}
        \item \TODO
    \end{itemize}

    \item Jiří Čechák
    \begin{itemize}
        \item application frontend
        \item application design (user interface)
        \item \TODO
    \end{itemize}

    \item Tadeáš Pavlík
    \begin{itemize}
        \item \TODO
    \end{itemize}

    \item Václav Stehlík
    \begin{itemize}
        \item application backend
        \item backend and frontend linking
        \item \TODO
    \end{itemize}
\end{itemize}

\section{Application initialization}

Firstly, we have chosen a stable, reliable and safe platform supporting team project development - GitHub. Furthermore, GitHub provides a version control system management and the integration with IBM Cloud is supported. Subsequently, after the repository was configured properly, we had to choose the technologies. We decided to use \#C (general-purpose, multi-paradigm and object oriented programming language) and React (library for building user interfaces) for the project. Based on the above, we have initialized the project skeleton - a \textit{'Hello, World!'} application.

\section{Application implementation}

\TODO

\section{Application deployment}

\TODO

\section{Application instructions}

\TODO

\section{Screenshots}

\TODO



\begin{comment}

\cite{travis-ci}

\newpage

\section{References}
\bibliographystyle{englishiso}
\begin{flushleft}
    \bibliography{quotation}
\end{flushleft}

\end{comment}

\end{document}
